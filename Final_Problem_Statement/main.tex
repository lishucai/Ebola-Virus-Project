\documentclass{article}
\usepackage[utf8]{inputenc}

\title{%
  Problem Statement \\
  \vspace{0.4cm}
  \large Core Body Temperature Estimation to Detect Ebola Virus Disease \\
  \vspace{0.4cm}
  \large CS 461, Fall 2017, Group 34\\
    }
\author{Claude Maimon,  Brian Lee Huang , and Bianca Beauchamp}
\date{ 20 October}

\begin{document}

\maketitle


\begin{abstract}
Currently, to check for Ebola, doctors must use contact sensors such as thermometers to test for an elevated core body temperature in patients. This process is slow and can potentially lead to infection of staff members as well as other patients. With a large volume of patients such as the 2015 Ebola outbreak, this method is extremely inefficient and can contribute to the spread of the Ebola Virus. This project will aim to create a device that will be able to quickly estimate a person’s core body temperature from a distance using only stand-off sensors. To do this, we will construct a mathematical model that will use data taken from a thermal image camera. The mathematical model will determine whether or not a patient’s core body temperature is higher than normal. We will be working with Medecins Sans Frontieres and the MIME Capstone Team to complete this project.   \end{abstract}
\newpage
\section{Problem}
In the medical industry, those who care for patients must take a patient’s core body temperature to determine whether or not the patient is fighting an infection. Currently, there is an Ebola outbreak in West Africa and there are many health-care professions that are risking infection to help people in need of medical help. In these health center in West Africa, the only way to take a patient's temperature is by using a device that comes in contact with the patent. The proximity to the patient combined with the time it takes to get a temperature reading increases the risk of infection to the caretaker. Moreover, infected patients that are waiting to be checked puts other non-symptomatic patients at risk of infection. That’s because the healthy and the symptomatic patients are grouped together in one area. Without some sort of separation, the risk of spreading the Ebola virus increases as more and more people come in to get a checkup.
The doctors want to be able to separate non-symptomatic people and patient that have the Ebola virus symptoms into different areas. This approach quarantines those infected with Ebola and will lower the risk of infection for both staff and public members. 
\par 
There is a need for a system that will estimate a patient core body temperature with no human contact. A thermal camera can pick up increases in skin temperature, but there are many different factors that could increase the skin’s temperature.Solely relying on skin temperature can produce many false positives and can cause more problems than it solves. A computational model is needed to calculate the core body temperature using data from skin temperature that accounts for all the different factors that could affect the skin’s temperature. This system should also work with minimal risk for the staff and the public members. The faster an individual could be diagnosed, the safer it will be for the present individuals at the health center.
  \par


\section{Proposed solution}
Our client needs a solution that can detect individuals that are showing signs of Ebola quickly without doctor supervision or human contact. We will design a computational model that will predict patient’s core body temperature. Our mathematical model will process patients’ skin temperature to estimate their core body temperature. This model should be able to detect whether or not a person has high core body temperature.  Our model will use data from stand-off sensors, mainly thermal images. The thermal image will be obtained through a device much like a metal detector at an airport. Patients will walk through an entryway and a thermal image will be taken as they walk through, one at a time. \par
We will study the patterns of the data to create a precise mathematical model that will accurately predict an individual’s body core temperature. The model will return a clear signal of whether or not the patient has high core body temperature. According to the model signal, patients will be directed to the appropriate waiting area. This will help to separate patients with high body temperature from people with normal body temperatures. Symptomatic patients will be receiving additional testing.



\section{Performance metrics}

Our final model should be able to predict if a person high core body temperature just by using a thermal image of them with only a 10 percent false negative rate. The whole process of taking a patient’s temperature and analyzing the data should take less time than a normal checkup. While a thermometer check takes minutes to be operated, our prototype will take seconds to produce a reading. The device should also be very simple to use, providing a simple yes or no. Either the patient has high core body temperatures or not. Our device should not require any specialized training or knowledge to use, the readouts of the data should be simple enough that it should be blatantly obvious if a person has high core body temperature or not.

 

\end{document}
