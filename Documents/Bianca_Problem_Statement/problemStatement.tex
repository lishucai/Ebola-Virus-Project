\documentclass[10pt]{IEEEtran}

\begin{document}

\begin{titlepage}
\title{Problem Statement}
\author{Bianca Beauchamp}
\maketitle
\centering
Fall Term 2017\\CS461: Computer Science Senior Capstone
\section{Abstract}
There is currently no way to take a persons temperatue without being within a close proximity to their bodies. This increases the risk of infection to those caring for the patients. This project will aim to create a device that will be able to quickly take a persons core body temperature from a distance using only stand-off sensors.
\end{titlepage}

\section{Problem Definition}
In the medical indistry, those who care for patients typically have to take a patients core body temperature in order to determine wether or not the patient is fighting an infecion. Currently, the only way of taking a patients temperatue is by using a device that comes in contact with the patent. The close proximity to the patient combined with the time it takes to take a patients temperature increases the risk of infection to the care taker.\\
Currently there is an Ebola outbreak in West Africa and there are many health care professions there risking infection to help people. One of the vitals they are screning people for is temperatue because those who are symptomatic have been showing a high core body temperaute. Since there are so many sick people, these care takers are taking the temperatures of hundreds of people which greatly increases their risk of infection. 

\section{Proposed Solution}
 An automated triage system that is reliable would dramaticly reduce the risk of infection to caretakers. The most important component of a system like this would be the ability to take a patents temperature from a distance to determine which people are symptomatic and which ones are asymptomatic. The creation of a no contact devidce to estimate a patitents core body temperature could by done by using stand-off sensors combined with thermal imaging to create a mathmatical model to estimate core body temperatue.

\section{Preformance Metrics}

The final product should be a prototype that attempts to predict core body temperature. Accuracy will be measured by collecting data from a contact thermometer and from our device to obtain a percent accuracy. The goal is for us to reach at least 60 percent accuracy.

\end{document}