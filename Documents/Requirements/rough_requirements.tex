\documentclass{article}
\usepackage[utf8]{inputenc}

\title{%
  Problem Statement \\
  \vspace{0.4cm}
  \large Core Body Temperature Estimation to Detect Ebola Virus Disease \\
  \vspace{0.4cm}
  \large CS 461, Fall 2017, Group 34\\
    }
\author{Claude Maimon,  Brian Lee Huang , and Bianca Beauchamp}
\date{\today}

\begin{document}

\maketitle


\begin{abstract}

\end{abstract}

\newpage
\section{Requirements}
	\begin{enumerate}
		\item The end goal of this project is to end up with a research paper. The paper should outline the problem that the project is trying to solve, the steps taken to solve the problem and how successfully we were
		at solving the problem. The paper should also allow the project to be continued if a higher percent accuracy is required. The whole process should be explained in detail allowing whoever wants to continue the
		project to continue without any problem. The end goal of this paper is if this project worked or not.
		\item The program that we are creating should be able to process data from a thermal camera. It should be able to process the image and focus on the person's upper body and record the temperature from that area.
		\item The program should be able to compare the temperature to the public average and check whether or not a person is above the average.
		\item The whole process of taking the image and analyzing it should be faster than a regular check up.
		\item The program should be easy to use. Zero background knowledge should be required to operate it.
		\item The code will be able to cut out the person from the background.
		\item We should provide multiple different statistical analysis of the data and provide the analysis with the smallest error.
		\item We should be able to isolate the person in the image, and then isolate the persons head in the image.
		\item A graph should be generated from a single person's data.
		\item Ear temperature of the person should also be taken. We will then use this temperature to compare it to what our model predicted.
		\item A mathematical model should be created out of all the data that is collected. The model will then be used to predict a person's core temperature.
	\end{enumerate}
\section{Stretch Goals}
	\begin{enumerate}
		\item We will go out and collect data to feed into the model, and make it able to predict more accurately.
	\end{enumerate}
\end{document}
