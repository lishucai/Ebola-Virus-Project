\documentclass[10pt, draftclsnofoot, onecolumn]{IEEEtran}
\usepackage[utf8]{inputenc}
\usepackage{lscape}
\usepackage{pgfgantt}


\title{%
  Requirements \\
  \vspace{0.4cm}
  \large Core Body Temperature Estimation to Detect Ebola Virus Disease \\
  \vspace{0.4cm}
  \large CS 461, Fall 2017, Group 34\\
    }
\author{Claude Maimon, Brian Lee Huang, and Bianca Beauchamp}
\date{\today}

\begin{document}

\maketitle

\begin{abstract}
	The end goal of this project is to end up with a research paper. The paper should outline the problem that the project is trying to solve, the steps taken to solve the problem and how successfully we were
	at solving the problem. The paper should also allow the project to be continued if someone chooses to. The whole process should be explained in detail allowing whoever wants to continue the
	project to continue without any problem. The main body of this research paper will be about the program that we develop to predict a person's core body temperature. The program should first be able to
	extract data from a thermal image. The data of the image should come from the top half, focusing on the head. It will then interpret the data to create a mathematical model that uses the temperature of a 
	person's skin as data and analyzes that information and predicts what their core body temperature is. A high accuracy rate is not strictly required as that is not the point of the project, 
	the goal of the is to determine if this method will be effective to detect whether a person is symptomatic with Ebola. A high accuracy rate is a good indicator that a mathematical model
	is a good way to predict, where a low accuracy rate indicates that we should look for an alternative method.
\end{abstract}

\newpage
\section{Introduction}
	\subsection{Purpose}
	\subsection{Scope}
	\subsection{Definitions, Acronyms and Abbreviations}
	\subsection{References}
	\subsection{Overview}
\section{Overall Description}
	\subsection{Product Perspective}
	\subsection{Product Functions}
	\subsection{User Characteristics}
	\subsection{Constraints}
	\subsection{Assumptions and Dependencies}
		\begin{center}
			\begin{landscape}
			\begin{ganttchart}[
				vgrid,
				hgrid style/.style=red, 
				]{1}{24}
				\gantttitle{Ebola Virus Project Process}{24} \\
				\ganttbar{Processing thermal image}{1}{2} \\
				\ganttbar{Process data from pixels}{3}{5} \\
				\ganttbar{Collecting Data}{1}{5} \\
				\ganttbar{Analyzing data}{7}{10} \\
				\ganttbar{Program for graduate students}{12}{14} \\
				\ganttbar{Mathematical model}{12}{15} \\
				\ganttbar{Testing and result analyzing}{17}{18} \\
				\\[grid]
				\ganttbar{Stretch goals}{19}{20} \\
				\ganttlink{elem0}{elem1}
				\ganttlink{elem1}{elem3}
				\ganttlink{elem2}{elem3}
				\ganttlink{elem3}{elem4}
				\ganttlink{elem3}{elem5}
				\ganttlink{elem5}{elem6}
				\ganttlink{elem6}{elem7}
			\end{ganttchart}
			\end{landscape}
		\end{center}
\section{Specific requirements}
\section{Stretch Goals}
	\begin{enumerate}
		\item We will go out and collect data to feed into the model, and make it able to predict more accurately.
		\item Any accuracy rate that is above the baseline of 40\% accuracy.
	\end{enumerate}
\section*{Appendixes}
\section*{Index}
\end{document}
