\documentclass[onecolumn, draftclsnofoot,10pt, compsoc]{IEEEtran}
\usepackage[utf8]{inputenc}
\usepackage{lscape}
\usepackage{pgfgantt}
\usepackage{graphicx}
\usepackage{setspace}
\usepackage{url}
\usepackage{import}
%\usepackage{pdfpages}
%\usepackage{caption}
\usepackage{geometry}
\usepackage{listings}
\usepackage{color}

\definecolor{codegreen}{rgb}{0,0.6,0}
\definecolor{codegray}{rgb}{0.5,0.5,0.5}
\definecolor{codepurple}{rgb}{0.58,0,0.82}
\definecolor{backcolour}{rgb}{0.95,0.95,0.92}

\lstdefinestyle{mystyle}{
	backgroundcolor=\color{backcolour},   
	commentstyle=\color{codegreen},
	keywordstyle=\color{magenta},
	numberstyle=\tiny\color{codegray},
	stringstyle=\color{codepurple},
	basicstyle=\footnotesize,
	breakatwhitespace=false,         
	breaklines=true,                 
	captionpos=b,                    
	keepspaces=true,                 
	numbers=left,                    
	numbersep=5pt,                  
	showspaces=false,                
	showstringspaces=false,
	showtabs=false,                  
	tabsize=2
}

\lstset{style=mystyle}
\geometry{textheight=9.5in, textwidth=7in}

\def \CapstoneTeamName{		Ebola Team}
\def \CapstoneTeamNumber{		34}
\def \GroupMemberOne{			Claude Maimon}
\def \GroupMemberTwo{			Brian Lee Huang}
\def \GroupMemberThree{			Bianca Beauchamp}
\def \CapstoneProjectName{		Ebola Prediction Model}
\def \CapstoneSponsorCompany{	Professor Bill Smart}


\def \DocType{		Final Report
	%Requirements Document
	%Technology Review
	%Design Document
	%Progress Report
}

\newcommand{\NameSigPair}[1]{\par
	\makebox[2.75in][r]{#1} \hfil 	\makebox[3.25in]{\makebox[2.25in]{\hrulefill} \hfill		\makebox[.75in]{\hrulefill}}
	\par\vspace{-12pt} \textit{\tiny\noindent
		\makebox[2.75in]{} \hfil		\makebox[3.25in]{\makebox[2.25in][r]{Signature} \hfill	\makebox[.75in][r]{Date}}}}
% 3. If the document is not to be signed, uncomment the RENEWcommand below
\renewcommand{\NameSigPair}[1]{#1}

%%%%%%%%%%%%%%%%%%%%%%%%%%%%%%%%%%%%%%%
\begin{document}
	\begin{titlepage}
		\pagenumbering{gobble}
		\begin{singlespace}
			%\includegraphics[height=4cm]{coe_v_spot1}
			\hfill 
			% 4. If you have a logo, use this includegraphics command to put it on the coversheet.
			%\includegraphics[height=4cm]{CompanyLogo}   
			\par\vspace{.2in}
			\centering
			\scshape{
				\huge CS Capstone \DocType \par
				{\large\today}\par
				\vspace{.5in}
				\textbf{\Huge\CapstoneProjectName}\par
				\vfill
				{\large Prepared for}\par
				\Huge \CapstoneSponsorCompany\par
				\vspace{5pt}
				{\large Prepared by }\par
				Group\CapstoneTeamNumber\par
				% 5. comment out the line below this one if you do not wish to name your team
				\CapstoneTeamName\par 
				\vspace{5pt}
				{\Large
					\NameSigPair{\GroupMemberOne}\par
					\NameSigPair{\GroupMemberTwo}\par
					\NameSigPair{\GroupMemberThree}\par
				}
				\vspace{20pt}
			}
			\begin{abstract}
				ADD ABSTRACT	
			\end{abstract}  
		   
		\end{singlespace}
	\end{titlepage}

	\newpage
	\pagenumbering{arabic}
	\tableofcontents
	% 7. uncomment this (if applicable). Consider adding a page break.
	%\listoffigures
	%\listoftables
	\clearpage
	
	\section{Introduction}
	
	This project was requested by Bill Smart, who is working with NIH and Doctors without borders. It was requested to try and use another method for checking patients for Ebola. A new method is important, as Ebola is transferred through bodily fluids, so the less contact doctors and other patients have with each other the less likely it will spread. The team that worked on this project were Claude Maimon, Bianca Beachamp, and Brian Huang. Claude was responsible for working on the image analysis, Bianca was responsible for working on the model creation and testing, and Brian was responsible for the final production mode and user interface. For this project our client was Bill Smart, we had weekly check ins with him and he would provide advice and troubleshooting. We also worked with several graduate students along with our TA Andrew Emmott.
		
	\section{Requirements Document}
	\subsection{Old Requirement Document}
	\import{sections/}{requirments.tex}
	\subsection{New Requirements}
	There were two main things that we changed in the requirement documents. The first was the 40 percent success rate. From the initial research that we did, we found that some previous attempts of this approach produced about 40 percent success rate. When we started the project we thought that we could achieve that success rate. However, as we'll discuss in this report, we ran into many problems with the camera. The camera wasn't calibrated right and it returned bad values. We tried many things to fix it even though it is not in our requirements to try and fix the camera. When first writing this document, we assumed the camera would be working correctly. Since that wasn't the case, we had a hard time collecting good data. That's why we took out the 40 percent  success rate goal. Our client, Bill Smart, agreed with this change. Another change that we made was producing a report and not a research paper as a final product with the code. Our client agreed that there is no need for a research paper. We just needed to produce a report that will help his graduate students in continuing this project. 
	\subsection{Final Gantt Chart}
	\begin{center}
		\begin{landscape}
			\begin{ganttchart}[
				vgrid,
				x unit=0.75cm,
				y unit chart=1cm,
				hgrid style/.style=red
				]{24}{24}
				\gantttitle{Ebola Virus Project Process}{24} \\
				\ganttbar{Processing thermal image}{1}{15} \\
				\ganttbar{Process data from pixels}{2}{15} \\
				\ganttbar{Collecting Data}{12}{18} \\
				\ganttbar{Analyzing data}{12}{13} \\
				\ganttbar{Mathematical model}{13}{17} \\
				\ganttbar{Testing and result analyzing}{18}{19} \\
				\ganttbar{Program for graduate students}{15}{17} \\
				\\[grid]
				\ganttbar{Testing and result analyzing}{18}{19} \\
				
				\ganttbar{Stretch goals}{19}{20} \\
				\ganttlink{elem0}{elem1}
				\ganttlink{elem1}{elem3}
				\ganttlink{elem2}{elem3}
				\ganttlink{elem3}{elem4}
				\ganttlink{elem4}{elem5}
				\ganttlink{elem5}{elem6}
				\ganttlink{elem6}{elem7}
			\end{ganttchart}
		\end{landscape}
	\end{center}
	
	\section{Design Document}
	
	\subsection{Original Design Document}
	\import{sections/}{Design_Document.tex}
	\subsection{New Design Changes}
		The main change that we did in the design document is the same as the requirement change. We changed the final product from a research paper to a detailed report. Our client agreed that there is no need for a research paper. Instead, we produced a detailed report that can assist someone in continuing our project in the future. 
	
	\section{Tech Reviews}
	\subsection{Brian's Tech Review}
	\import{sections/}{Brian_Huang_Tech_Review.tex}
	\subsection{Claude's Tech Review}
	\import{sections/}{Claude_tech.tex}
	\subsection{Bianca's Tech Review}
	\import{sections/}{Binaca_tech.tex}
	
	\section{Weekly Blog Posts}
	
	\subsection{Brian}
	\import{sections/}{Brian_weekly.tex}
	
	\subsection{Claude}
	\import{sections/}{Claude_weekly.tex}
	
	\subsection{Bianca}
	\import{sections/}{Bianca_weekly.tex}
	\newpage
	\section{Final Poster}	
		Our final poster as it was presented at the Undergraduate Engineering Expo. 	
		\begin{figure}[!hb]
			\centering
			\includegraphics[width=\textwidth, height=20cm]{group34poster.eps}
			\caption{Our final poster}
		\end{figure}

	\section{Project Documentation}
	\import{sections/}{documentation.tex}
	
	\section{Recommended Technical Resources}
	
	\section{Conclusions and Reflections}
	
	\subsection{Brian}
	For this project I worked on the production mode, the user interface and the evaluation method. The production mode required the uses of a linux machine, so I learned how to install linux onto my computer. It was a lot of trial and error messing with the BIOS, but I finally got it to work. I also learned about the Receiver Operating Characteristic curve, what it does, what it looks for and how it works. This was used to create the evaluation method of the project. This piece is working, but it just needs data to be used. 
	
	After working on this project I learned that project work can be even more difficult than working alone, but if done correctly it can also be much easier. If the team communicates and cooperates with each other, the project can be done quickly and easily as there are more people thinking about the same problem, effectively doubling or tripling the problem solving power. However the key here is good communication and cooperation, if these are bad then the group work can become much more difficult than if worked on my a single person. The management of the project works exactly the same way, where communication and cooperation are extremely essential. Without communication one person ends up doing all of the work by themselves, where other group members may have been willing to share the load. Overall working in teams is definitely more difficult than working alone. It requires a lot of juggling of other group members workload and schedules. It becomes a balancing act of trying to keep everyone as happy as possible, as having a disgruntled team member can cause many problems.
	
	If I could do this whole project over again I would definitely start everything earlier than I did. Although everything got done on time, starting everything earlier would help reduce the stress we had over the three terms, as we would have more room to work on it. 
	
	\subsection{Claude}
	
	\subsection{Biance}
	
	
	\section{Appendix 1: Essential Code Listings}
	%	\import{sections/}{Appendix_1.tex}
	\section{Appendix 2}
	
	\newpage
	
	
	\bibliographystyle{IEEEtran}
	\bibliography{mybib}
	
\end{document}
