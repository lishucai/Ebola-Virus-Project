\begin{lstlisting}[language=Python, caption=Correcting the values from the CSV file]
	def correcting_values(temperatureData):
	#setting initial values so everything would be smaller
	min_value = 1000
	#goes through the list to find the absolute minimum
	for elements in temperatureData:
		temp_min_value = min(elements)
		if temp_min_value < min_value:
			min_value = temp_min_value
	# min_value = min(temperatureData)
	if DEBUG:
		print min_value
	#subtract the minimum value from all values
	for row_counter, elements in enumerate(temperatureData):
		for column_counter, element in enumerate(elements):
			element = element - min_value
			temperatureData[row_counter][column_counter] = element
			# [float(i)-min_value for i in elements]
			# print temperatureData[row][column]
	#returns the new list with changed values
	return temperatureData

\end{lstlisting}

\begin{lstlisting}[language=Python, caption=Finding the persons head in the picture]
def starting_point(temperatureData):
for row_counter, elements in enumerate(temperatureData):
for column_counter, element in enumerate(elements):
if element < 45 and element > 30:
good = []
for iterator in range(1, 5):
# print len(elements)
if (column_counter + iterator) < len(elements):
if elements[column_counter + iterator] < 45 and elements[column_counter + iterator] > 30:
# print "added to good"
good.append(iterator)
# print len(good)
if len(good) == 4:
# print "in"
coordinates = []
coordinates.append(row_counter)
coordinates.append(column_counter)
return coordinates
return []
\end{lstlisting}


\begin{lstlisting}[language=Python, caption=Claculating the mean temperature value]
def get_mean(x, y, WIDTH, HEIGHT, temperatureData):
if ((y + HEIGHT) > 120 or ((x + (WIDTH / 2)) > 160)):
print "PERSON OUT OF RANGE"
return 0
for row in range(y, (y + HEIGHT)):
# y is the row it begins, h is the width of the length of the head
for column in range(x - (WIDTH / 2), x + (WIDTH / 2)):
if (temperatureData[row][column] > 43 or temperatureData[row][column] < 35):
continue
else:
good_range.append(temperatureData[row][column])
if (len(good_range) == 0):
return 0
range_mean = sum(good_range) / float(len(good_range))
return range_mean
\end{lstlisting}



\begin{lstlisting}[language=Python, caption=Prodiction Mode]
def takeThermalImage(configFile):
path = "./Images"
name = "image"
with IrCamera(configFile) as ir_cam:

for i in range(40):
data_t, data_p = ir_cam.get_frame()
cv2.imshow('visual data', data_p)
print '----------------'
print data_t
print '----------------'
cv2.waitKey(5)

n_rows, n_cols = data_t.shape
# print data_t[0]

if (i > 10):
filename = str(name + str(i) + ".csv")
with open(os.path.join(path, filename), 'w') as csvfile:
thermalwriter = csv.writer(csvfile)
for row in range(n_rows):
thermalwriter.writerow(data_t[row]) 
\end{lstlisting}


\begin{lstlisting}[language=Python, caption=Taking an image and returning the mean]
config = sys.argv[1]
modelData = sys.argv[2]

# Taking the image using the first config file.
takeThermalImage(config)

# Getting the means of all the files.
imageMeans = images2.processImages()

if (0 in imageMeans):
print "An error has occurred. Try again."
else:
# Getting the mean of all the means
totalMean = mean(imageMeans)

# Running the total mean through the model.
print "Raw Processed Temperature: ", totalMean
print "Predicted Core Body Temperature: ", modelPredict(totalMean, modelData)

print "DONE"
\end{lstlisting}

